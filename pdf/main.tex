\documentclass[12pt]{article}

\input{./header}

\usepackage[a4paper,portrait,left=2cm,right=2cm,top=1.5cm,bottom=2.0cm]{geometry}

% footer
\usepackage{fancyhdr}

% table
\usepackage{array}
\setlength{\arrayrulewidth}{0.5mm}
\setlength{\tabcolsep}{2mm}
\renewcommand{\arraystretch}{1.2}

% footer
\thispagestyle{fancy}
\fancyhf{}
\renewcommand{\headrulewidth}{0pt}
\rfoot{\textsf{Josselin SCOUARNEC, 2022}}    
\cfoot{}
\lfoot{}


\begin{document}

% \pic{qrcode.png}
\sffamily

\begin{center}
    \LARGE\textbf{Comment fonctionne un ordinateur ?} \\
    \large Exemple du Little Man Computer.
\end{center}

\normalsize
Le Little Man Computer est un modèle éducatif d'ordinateur qui a été inventé dans les années 60 par le Dr. Stuart Madnick.
Il s'agit d'une pièce dans laquelle un petit bonhomme est enfermé avec 100 boîtes aux lettres
numérotées de 0 à 99 et quatre boîtes spéciales marquées \enquote{Entrée}, \enquote{Sortie}, \enquote{Numéro d'instruction} et \enquote{Accumulateur}.
Le numéro d'une boîte est ce qu'on appelle sont adresse.
\par
Pour utiliser l'ordinateur, le petit bonhomme a à disposition un tableau avec des instruction.
D'abord, il prend le nombre qui se trouve dans la boîte dont le numéro est indiqué par le compteur d'instruction.
Ensuite il regarde dans le tableau l'action qu'il doit effectuer.
Enfin, si ce n'était pas l'instruction Arrêt ou Saut, il passe à l'instruction suivante en ajoutant 1 au compteur d'instruction.


\vfill\par
Un simulateur de Little Man Computer est disponible sur mon site web :

\begin{center}
\includegraphics[width=4.0cm]{figures/qrcode.png}\\
\texttt{https://cpu.scouarn.fr}
\end{center}


\vfill


\begin{center}
\resizebox{\linewidth}{!}{
\begin{tabular}{| c | c | c | m{10cm} |}
    \hline 
    \textbf{Code} & \textbf{Nom} & \textbf{Mnémonique} & \textbf{Action}  \\
    \hline 
    000       & Arrêt           & HLT & Arrêter l'exécution du programme, le numéro d'instruction n'est pas incrémenté. \\
    \hline 001       & Entrée          & INP & Prendre le nombre qui est dans la boîte d'entrée et le mettre dans l'accumulateur. \\
    \hline 002       & Sortie          & OUT & Prendre le nombre qui est dans l'accumulateur dans le mettre dans boîte de sortie. \\
    \hline 100 à 199 & Sauvegarde      & STA & Remplacer la valeur à l'adresse formée par les deux derniers chiffres du code par le contenu de l'accumulateur. \\
    \hline 200 à 299 & Chargement      & LDA & Remplacer le contenu de l'accumulateur par la valeur à l'adresse formée par les deux derniers chiffres du code. \\
    \hline 300 à 399 & Addition        & ADD & Ajouter à l'accumulateur la valeur à l'adresse formée par les deux derniers chiffres du code. \\
    \hline 400 à 499 & Soustraction    & SUB & Soustraire à l'accumulateur la valeur à l'adresse formée par les deux derniers chiffres du code. \\
    \hline 500 à 599 & Saut            & BRA & Changer le numéro d'instruction par le nombre formée par les deux derniers chiffres du code. \\
    \hline 600 à 699 & Saut si zéro    & BRZ & Si la valeur de l'accumulateur est égale à zéro, changer le numéro d'instruction par le nombre formée par les deux derniers chiffres du code. \\
    \hline 700 à 799 & Saut si positif & BRP & Si la valeur de l'accumulateur est supérieure ou égale à zéro, changer le numéro d'instruction par le nombre formée par les deux derniers chiffres du code. \\
    \hline 800 à 899 & Saut si négatif & BRN & Si la valeur de l'accumulateur est strictement inférieure à zéro, changer le numéro d'instruction par le nombre formée par les deux derniers chiffres du code. \\
    \hline 
\end{tabular}}
\end{center}

\end{document}
